\documentclass[11pt,a4paper]{article}
\usepackage[english]{babel}					% Use english
\usepackage[utf8]{inputenc}					% Caracteres UTF-8
\usepackage{graphicx}						% Imagenes
\usepackage[hidelinks]{hyperref}			% Poner enlaces sin marcarlos en rojo
\usepackage{fancyhdr}						% Modificar encabezados y pies de pagina
\usepackage{float}							% Insertar figuras
\usepackage[textwidth=390pt]{geometry}		% Anchura de la pagina
\usepackage[nottoc]{tocbibind}				% Referencias (no incluir num pagina indice en Indice)
\usepackage{enumitem}						% Permitir enumerate con distintos simbolos
\usepackage[T1]{fontenc}					% Usar textsc en sections
\usepackage{amsmath}						% Símbolos matemáticos

% Comando para poner el nombre de la asignatura
\newcommand{\subject}{Natural Language Processing}
\newcommand{\autor}{Vladislav Nikolov Vasilev}
\newcommand{\titulo}{Assignment 1}
\newcommand{\subtitulo}{Quora challenge}
\newcommand{\masters}{Master in Fundamental Principles of Data Science}


% Configuracion de encabezados y pies de pagina
% \pagestyle{fancy}
% \lhead{}
% \rhead{\subject{}}
% \lfoot{\masters}
% \cfoot{}
% \rfoot{\thepage}
% \renewcommand{\headrulewidth}{0.4pt}		% Linea cabeza de pagina
% \renewcommand{\footrulewidth}{0.4pt}		% Linea pie de pagina

\begin{document}
\pagenumbering{gobble}

% Title page
\begin{titlepage}
  \begin{minipage}{\textwidth}
    \centering
    \includegraphics[scale=0.25]{img/ub-logo}\\[2cm]
    
    \textsc{\Large \subject\\[0.5cm]}
    \textsc{\uppercase\expandafter{\masters}}\\[1.5cm]
    
    \noindent\rule[-1ex]{\textwidth}{1pt}\\[1.5ex]
    \textsc{{\Huge \titulo\\[0.5ex]}}
    \textsc{{\Large \subtitulo\\}}
    \noindent\rule[-1ex]{\textwidth}{2pt}\\[3.5ex]
  \end{minipage}
  
  \vspace{2cm}
  
  \begin{minipage}{\textwidth}
    \centering
    
    \includegraphics[scale=0.4]{img/ub-ds-logo}
    \vspace{2cm}
    
    \textbf{Authors}\\ {Irene Bonafonte Pardàs}\\{Otis Carpay}\\{Vladislav Nikolov Vasilev}\\[2.5ex]
    \textsc{Faculty of Mathematics and Computer Science}\\
    \vspace{1em}
    \textsc{Academic year 2021-2022}
  \end{minipage}
\end{titlepage}

\pagenumbering{arabic}
\tableofcontents
\thispagestyle{empty}				% No usar estilo en la pagina de indice

\newpage

\section{Introduction}

In this assignment we are going to try to solve the Quora Question Pairs challenge. Given a pair
of questions, we have to automatically determine whether they are semantically equivalent or not.
The goal of this is to reduce the number of duplicate questions and improve the overall user experience.

In order to solve this challenge, we are fist going to try a simple solution which will allow us
to get a better understanding of the problem and identify possible flaws. After that, we are
going to refine this initial solution in hopes of obtaining a model that is more robust and better
able to identify duplicate questions.

\section{Simple solution}

\section{Improved feature vectors and distances}

\subsection{Irene's features}

\subsection{Otis' features}

\subsection{Vladislav's features}

\section{Final results}

\newpage

\begin{thebibliography}{5}

\bibitem{nombre-referencia}
Texto referencia
\\\url{https://url.referencia.com}

\end{thebibliography}

\end{document}

